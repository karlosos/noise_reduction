\documentclass[wi]{zut}

\usepackage{microtype}
\usepackage{wrapfig}
\usepackage{lipsum}
\author{Karol Działowski \\ Marcin Łukasik}
\title{Usuwanie szumów z sygnałów mowy}
\przedmiot{Sygnały Akustyczne}

\makemetadata

\begin{document}

\maketitle

\begin{abstract}
    Lorem ipsum dolor sit amet, consectetur adipiscing elit, sed do eiusmod tempor incididunt ut labore et dolore magna aliqua. Ut enim ad minim veniam, quis nostrud exercitation ullamco laboris nisi ut aliquip ex ea commodo consequat. Duis aute irure dolor in reprehenderit in voluptate velit esse cillum dolore eu fugiat nulla pariatur. Excepteur sint occaecat cupidatat non proident, sunt in culpa qui officia deserunt mollit anim id est laborum.
\end{abstract}
\tableofcontents

\section{Wstęp}

\lipsum{1-2}


\begin{figure}[H]
    \centering
    \begin{subfigure}{0.49\textwidth}
    \includegraphics[width=1\linewidth]{graphic/logo_WE.pdf}
    \caption{Logo WE}
    \end{subfigure}
    \begin{subfigure}{0.49\textwidth}
    \includegraphics[width=1\linewidth]{graphic/logo_WI.pdf}
    \caption{Logo WI}
    \end{subfigure}
    \caption{Wraz ze zwiększeniem parametru $\alpha$ poświata auta jest krótsza ale bardziej wyraźna.}
    \label{fig:spectograms}
    \source{Opracowanie własne}
\end{figure}

\begin{equation}
    S P(x, y)=\left\{\begin{array}{l}
1 \text { if } \alpha \leq \frac{I^{v}(x, y)}{B^{v}(x, y)} \leq \beta \wedge\left(I^{S}(x, y)-B^{S}(x, y)\right) \leq \tau_{S} \\
\wedge\left|I^{i l}(x, y)-B^{h \prime}(x, y)\right| \leq \tau_{h} \\
0 \text { otherwise }
\end{array}\right.
\end{equation}

\code{Implementacja usuwania cieni}
{Opracowanie własne}{\label{kod:przyklad}}
\begin{lstlisting}[language=Python]
I = cv2.cvtColor(frame, cv2.COLOR_BGR2HSV)
B = np.mean(last_frames, axis=0).astype('uint8')
a = 1
b = 3
th = 250
ts = 250
cond_1 = I[:, :, 2]/B[:, :, 2] >= a
cond_2 = I[:, :, 2]/B[:, :, 2] <= b
SP_1 = np.logical_and(cond_1, cond_2)
cond_3 = I[:, :, 1] - B[:, :, 1] <= ts
cond_4 = np.abs(I[:, :, 0] - B[:, :, 0]) <= th
SP_2 = np.logical_and(cond_3, cond_4)
SP = np.logical_and(SP_1, SP_2)
SP = np.logical_not(SP)
SP = SP.astype('uint8')*255
\end{lstlisting}

Nie udało mi się znaleźć parametrów, które dobrze by wykrywały cienie na badanym materiale. Przykład najlepszych rezultatów przedstawiono poniżej:

\section{Przygotowanie danych}

\section{Model}

\section{Testy}

\section{Podsumowanie}

W ramach zajęć przetestowano metody modelowania tła. Zaimplementowania metodę usuwania tła z wykorzystaniem ostatnich kilkunastu klatek z wykładniczym ważeniem klatek. Porównano uzyskane wyniki z metodą HOG2. Zaimplementowano metodę modelowania cieni na obrazie z wykorzystaniem przestrzeni barw HSV \cite{granovetter1978threshold}.


\printbibliography[heading=bibintoc]

\appendix

% \section{Przykłady do kopiowania}

% \begin{table}[H]
% \caption{Czas trwania jednej epoki dla rozmiaru obrazu}
% \vspace{1em}
% \centering
% \begin{tabular}{@{}lr@{}}
% \toprule
% Rozmiar          & Czas {[}s{]} \\ \midrule
% $32 \times 32$   & 0.1288       \\
% $64 \times 64$   & 0.2000       \\
% $128 \times 128$ & 1.046        \\ \bottomrule
% \end{tabular}
% \end{table}

% \code{Przykład kodu}
% {Opracowanie własne}{\label{kod:przyklad}}
% \begin{lstlisting}[language=Python]
% discriminator = make_discriminator_model()
% generator = make_generator_model()
% \end{lstlisting}

% \begin{figure}[H]
%     \centering
%     \includegraphics[width=0.7\linewidth]{images/cfa_1.png}
%     %\vspace{1em}
%     \caption{Przykładowy obrazek}
%     \label{fig:pdgd}
%     \source{Opracowanie własne}
% \end{figure}


\end{document}
